\section{Introduction}

OpenMath~\cite {BusCapCar:oms04} is a semantic representation format of mathematical
objects and formulae. In a nutshell, OpenMath standardizes six basic object types
(symbols, variables, numbers, strings, and foreign objects), three ways of building
complex objects: (function) application, binding, and management facilities like structure
sharing and error reporting. The OpenMath object model underlies Content
MathML~\cite{CarlisleEd:MathML3:on}, making it well-integrated with MathML presentation

There are several encodings of OpenMath Objects, most notably the XML and Binary
encodings. In this paper we propose another one based on JSON a lightweight
data-interchange format that used heavily in the Web Applications arena.

JSON~\cite{JSON:web}, short for \textbf{J}ava\textbf{S}cript \textbf{O}bject
\textbf{N}otation, is a lightweight data-interchange format.  While being a subset of
JavaScript, it is defined independently.  JSON can represent both primitive types and
composite types.

Primitive JSON data types are strings (e.g. \lstinline{"Hello world"}), Numbers
(e.g. \lstinline{42} or \lstinline{3.14159265}), Booleans (\lstinline{true} and
\lstinline{false}) and \lstinline{null}.  Composite JSON types are either
(non-homogeneous) arrays (e.g. \lstinline{[1, "two", false]}) or key-value pairs called
objects (e.g. \lstinline|{"foo": "bar", "answer": 42}|).

Constructs corresponding to JSON objects are found in most programming languages.
Furthermore, the syntax is very simple; hence many languages have built-in facilities for
translating their existing data structures to and from JSON.  The use for an OpenMath JSON
encoding is clear: It would enable easy use of OpenMath across many languages.

In the next Section we survey two existing OpenMath JSON
encodings. Section~\ref{sec:encoding} proposes a new encoding that combines the advantages
and alleviates their disadvantages. We give a thorough specification of the encoding,
present a JSON schema implemented in TypeScript, and provide a web service that validates
JSON-encoded OpenMath and transforms OpenMath objects between XML and JSON
encodings. Section~\ref{sec:concl} concludes the paper.

%%% Local Variables:
%%% mode: latex
%%% TeX-master: "paper"
%%% End:

%  LocalWords:  standardizes textbf textbf cript textbf bject otation sec:concl
