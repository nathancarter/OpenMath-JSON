\section{The OpenMath-JSON encoding}

To formalize our encoding, we decided to use JSON Schema~\cite{handrewsjsonschema:18}. 
This defines a vocabulary allowing us to validate and annotate JSON documents. 
Additionally, tools for programatic verification exist in many languages. 

Unfortunatly, JSON schema is often tedious to write and read for humans. 
This is especially true when it comes to recursivly defined data strucutures.
OpenMath has many recusrive structures.
Instead of writing our encoding in JSON Schema directly, we decided to write the schema in a different language and then compile it to JSON Schema. 

For this purpose, we decided to make use of TypeScript~\cite{typescript:webpage}. 
TypeScript is a language derived from JavaScript -- TypeScript files are JavaScript plus type annotations. 
As such, it can be easily written and understood by humans. 
On top of typescript, we make use of a compiler~\cite{vega-ts-jscon-schema-generator:webpage} from TypeScript definitions into JSON Schema. 

\subsection{Encoding Details}

In general, objects in our encoding look similar to the following:
\begin{lstlisting}
{
    "kind": "OMV",
    "id": "something",
    "name": "x"
}
\end{lstlisting}

The \texttt{kind} attribute specifies which kind of OpenMath object this is. 
These values correspond to the element names used in the XML encoding. 
This correspondence lays the foundations of easy translation between the two. 
In TypeScript this property is also refered to as a \textit{Type Guard}, because if \textit{guards} the type of object that is represented. 

Like in the XML encoding it is possible to make use of structure sharing. 
For this purpose the \texttt{id} attribute can be used. 
We will come back to this in more detail below, when we define to the \texttt{OMR} type.

In the following we will go over the details of our encoding. 
For this we will make use of a TypeScript-like syntax, that is easily readable. 
In our description we omit the \texttt{id} attribute, which can be added to any encoded object. 
The complete source code of our encoding -- and details on how to use it -- can be found on Github~\cite{URL:openmathjson:github}. 

\subsubsection{Object Constructor -- OMOBJ}

The OpenMath Object Constructor -- \texttt{OMOBJ} -- is defined as follows:
\begin{lstlisting}
{
    "kind": "OMOBJ",
    /** optional version of openmath being used */
    "openmath": "2.0",
    /** the actual object */
    "object": omel /* any element */
}
\end{lstlisting}
Concretly, the integer 3 encapsulated in an object constructor using this encoding is as follows:
\begin{lstlisting}
{
    "kind": "OMOBJ",
    "openmath": "2.0",
    "object": {
        "kind": "OMI", 
        "integer": 3
    }
}
\end{lstlisting}

Let us have a look at this first example attribute for attribute. 

The first attribute -- \texttt{kind} -- represents the type of OpenMath object in question. 
Notice that it occurs twice -- once in the \texttt{OMOBJ} and a second time in the wrapped \texttt{OMI}. 
We will talk in detail about integer representation below, and hence only care about this first one. 

The second attribute -- \texttt{openmath} -- is defined as optional by our schema. 
This indicates the version of OpenMath that is being used -- \lstinline{"2.0"} in our case. 

The third and final attribute is the \texttt{object} attribute. 
This contains the wrapped object -- it is defined as of \texttt{omel} type. \ednote{Do we want to keep this order?}
This type \texttt{omel} can contain any OpenMath element -- concretly
primitive objects (Bytes \texttt{OMB}, Integers \texttt{OMI}, Floats \texttt{OMF}, Strings \texttt{OMSTR}, Symbols \texttt{OMS}, Variables \texttt{OMV}), 
complex elements (Application \texttt{OMA}, Binding \texttt{OMBIND}, Attribution \texttt{OMATTR}) or
Errors \texttt{OME} and References \texttt{OMR}. 
In this particular case, we just have the integer $3$. 

\subsection{The Demo Site}

To demonstrate our OpenMath-JSON encoding, we have created a demo site which can be found at~\cite{openmathjson:web}. 
This site is implemented in TypeScript and encapsulated using Docker\cite{docker:webpage}. 

The demo site serves three purposes.

Primarily, it serves as a presentation of the encoding, providing examples and documenting it's usage. 

Secondly, it enables validation of OpenMath JSON objects. This can be seen in Figure\ednote{Make figure}.
The user can enter some JSON, press the \textit{Validate JSON} button, and receive immediate feedback if their JSON is a valid OpenMath object or not. 
In particular, the user can also see a detailed error message if their object is not valid OpenMath JSON. 

This makes use of the OpenMath JSON schema, and validates the users' JSON using a generic JSON Schema Validator. 
Furthermore, this is also exposed using a REST API, enabling easy validation of OpenMath JSON in other applications. 

Thirdly, it enables translation between XML and JSON OpenMath objects. 
Like for validation, the site enables the user to enter some JSON and be presented with some XML and vice-versa. 
This can be seen in Figure\ednote{Make figure + screenshot}.

As we designed our encoding with this translatability goal in mind, the implementation of it was straight-forward. 
Nonetheless, this translation is also exposed using a REST API. 