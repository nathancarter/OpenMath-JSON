\documentclass[usenames,dvipsnames]{beamer}

\usepackage{mathtools} % for over- and underbrackets

% listings syntax highlighting for JSOn
\usepackage{listings}

\colorlet{bool}{red}
\colorlet{string}{blue}
\colorlet{number}{orange}
\colorlet{comment}{OliveGreen}
\colorlet{type}{purple}

\lstset{
    basicstyle=\small\ttfamily,
    columns=fullflexible,
    % strings
    string=[s]{"}{"},
    stringstyle=\color{string},
    % booleans
    keywords={true,false},
    keywordstyle=\color{bool},
    % comments
    comment=[s]{/*}{*/},
    morecomment=[l]{//},
    commentstyle=\color{comment},
    % types
    morekeywords=[2]{base64string,omel,uri,name,any,string,integer,decimalInteger,hexInteger,float,decimalFloat,hexFloat,byte,OMFOREIGN,OMV,attvar},
    keywordstyle=[2]{\color{type}},
}

% digits
\lstset{literate=%
   *{0}{{{\color{number}0}}}1
    {1}{{{\color{number}1}}}1
    {2}{{{\color{number}2}}}1
    {3}{{{\color{number}3}}}1
    {4}{{{\color{number}4}}}1
    {5}{{{\color{number}5}}}1
    {6}{{{\color{number}6}}}1
    {7}{{{\color{number}7}}}1
    {8}{{{\color{number}8}}}1
    {9}{{{\color{number}9}}}1
    {e-}{{{\color{number}e-}}}2
    {-}{{{\color{number}-}}}1
    {.}{{{\color{number}.}}}1
    {base64string}{{{\color{type}base64string}}}{12}%keyword gets caught by number, so escape it
}

% lstinline will be of normal size
\makeatletter
\let\lstinline@org\lstinline
\def\lstinline{\lstinline@org[basicstyle=\ttfamily]}
\makeatother

\begin{document}

% TODO: Add CICM workshop details etc
\title{A Proposal for an OpenMath JSON Encoding}   
\author{Tom Wiesing\and\underline{Michael Kohlhase}} 
\date{\today} 

\begin{frame}
    \titlepage
\end{frame}

% =================================================================
% Motivation
% =================================================================
\begin{frame}[fragile]
    \frametitle{What is JSON?}
    \begin{itemize}
        \item JSON = \textbf{J}ava\textbf{S}cript \textbf{O}bject \textbf{N}otation%json.org
        \begin{itemize}
            \item lightweight data-interchange format
            \item subset of JavaScript (used a lot on the web)
            \item defined independently
        \end{itemize}
        \item Primitive types
        \begin{itemize}
                \item Strings (e.g. \lstinline{"Hello world"})
                \item Numbers (e.g. \lstinline{42} or \lstinline{3.14159265})
                \item Booleans (\lstinline{true} and \lstinline{false})
                \item \lstinline{null}
        \end{itemize}
        \item Composite types
        \begin{itemize}
            \item Arrays (e.g. \lstinline{[1, "two", false]})
            \item Objects (e.g. \lstinline|{"foo": "bar", "answer": 42}|)
        \end{itemize}
    \end{itemize}
\end{frame}

\begin{frame}
    \frametitle{Why an OpenMath encoding for JSON?}
    \begin{itemize}
        \item an OpenMath JSON encoding would make it easy to use across many languages
        \begin{itemize}
            \item JSON support exists in most modern programming languages
            \begin{itemize}
                \item corresponding native types common
                \item serialization to/from JSON without external library
            \end{itemize}
        \end{itemize}
        \item some existing approaches for an OpenMath JSON encoding
        \begin{itemize}
            \item discussed / suggested on the OpenMath mailing list
            \item we will look at two examples here
        \end{itemize}
    \end{itemize}
\end{frame}

\begin{frame}[fragile]
    \frametitle{XML as JSON}
    \begin{itemize}
        \item \textbf{Idea:} Generically encode XML as JSON
        \item use the JSONML standard for this%http://www.jsonml.org/
        \item e.g. $\mathrm{plus}(x, 5)$ corresponds to:
\begin{lstlisting}
[
    "OMOBJ",
    {"xmlns":"http://www.openmath.org/OpenMath"},
    [
        "OMA",
        ["OMS", {"cd": "arith1", "name": "plus"}],
        ["OMV", {"name": "x"}],
        ["OMI", "5"]
    ]
]
\end{lstlisting}
    \end{itemize}
\end{frame}

\begin{frame}[fragile]
    \frametitle{XML as JSON (2)}
    \begin{itemize}
        \item Advantages
        \begin{itemize}
            \item based on well-known XML encoding
            \item easy to understand based on it
        \end{itemize} 
        \item does not make use of JSON structures
        \begin{itemize}
            \item all attributes are encoded as strings, even numbers
            \item e.g. \lstinline{1e-10} (a valid JSON literal) can not be used
        \end{itemize}
        \item retains some of the XML akwardness
        \begin{itemize}
            \item introduces unnecessary overhead
            \item e.g. some pseudo-elements (such as OMATP) are needed
        \end{itemize}
    \end{itemize}
\end{frame}


\begin{frame}[fragile]
    \frametitle{OpenMath-JS}%https://github.com/lurchmath/openmath-js
    \begin{itemize}
        \item OpenMath-JS
        \begin{itemize}
            \item an (incomplete) implementation of OpenMath in JavaScript
            \item developed by Nathan Carter for use with Lurch Math on the web
            \item written in literate coffee script, a derivative language of JavaScript
        \end{itemize}
        \item e.g. $\mathrm{plus}(x, 5)$ corresponds to:
\begin{lstlisting}
{
    "t": "a",
    "c": [  
      {"t": "sy", "cd": "arith1", "n": "plus"}, 
      {"t": "v", "n": "x"}, 
      {"t": "i", "v": "5"}
    ]
}
\end{lstlisting}
    \end{itemize}
\end{frame}

\begin{frame}[fragile]
    \frametitle{OpenMath-JS (2)}
    \begin{itemize}
        \item does make use of JSON native structures
        \begin{itemize}
            \item much better than \textit{JSON-ML}
            \item small property names keep size of transmitted objects small
        \end{itemize}
        \item comes with some problems
        \begin{itemize}
            \item hard to read for humans
            \item written for \textit{JavaScript}, not JSON
            \item no formal schema
        \end{itemize}
    \end{itemize}
\end{frame}


% =================================================================
% Our Approach for an OpenMath encoding
% =================================================================

\begin{frame}
    \frametitle{Towards an OpenMath JSON Formalization}
    \begin{itemize}
        \item we need to write a new OpenMath JSON encoding
        \begin{itemize}
            \item combine advantages of the above two
            \item should be close to the XML encoding
            \item should make use of JSON concepts
        \end{itemize}
        \item we want to formalize this JSON encoding
        \begin{itemize}
            \item to verify JSON objects
            \item not done by existing approaches
        \end{itemize}
        \item comes with some positive side effects
        \begin{itemize}
            \item formalization of JSON $\Rightarrow$ structure definition in most languages
            \item trivial to use advanced serialization tools
            \begin{itemize}
                \item e.g. \textit{Protocol Buffers}, \textit{ZeroMQ}
            \end{itemize}
        \end{itemize}
        \item we can use JSON Schema%http://json-schema.org/
        \begin{itemize}
            \item a vocabulary allowing us to validate and annotate JSON documents
            \item tools for verification exist
        \end{itemize}
    \end{itemize}
\end{frame}

\begin{frame}
    \frametitle{Towards an OpenMath JSON Formalization (2)}
    \begin{itemize}
        \item JSON schema is often tedious to write and read
        \begin{itemize}
            \item especially when it comes to recusrive data types
            \item but implementation of it still exist 
        \end{itemize}
        \item \textbf{Idea:} Write schema in a TypeScript, compile into a JSON schema
        \begin{itemize}
            \item TypeScript = JavaScript + Type Annotations
            \item easily writeable and understandable
            \item a compiler from TypeScript Definitions into JSON Schema exists
        \end{itemize}
        \item We have done this, and will present some examples in the following slides
    \end{itemize}
\end{frame}

\begin{frame}
    \frametitle{Towards an OpenMath JSON Formalization (3)}
    \begin{itemize}
        \item We wrote a JSON Schema
        \begin{itemize}
            \item was written as described above
            \item we will give an overview how this looks below
        \end{itemize}
        \item We also wrote a translator from OpenMath XML to JSON
        \begin{itemize}
            \item a RESTful interface as part of MMT% TODO: Should we mention what MMT is
            \item was quick to implement given an existing XML implementation
        \end{itemize}
    \end{itemize}
\end{frame}

% =================================================================
% OpenMath JSON Overview
% =================================================================

\begin{frame}[fragile]
    \frametitle{General Structure of OpenMath objects}
    \begin{itemize}
        \item represent each OM Object as a Hashmap:
\begin{lstlisting}
{
    "kind": "OMV",
    "id": "something",
    "name": "x"
}
\end{lstlisting}
        \item \texttt{kind} attribute specifies the type
        \begin{itemize}
            \item called a \textit{type guard} in TypeScript
            \item has the same names as elements in the XML encoding
        \end{itemize}
        \item \texttt{id} attribute used for structure sharing
        \begin{itemize}
            \item like in xml
            \item referenced using \texttt{OMR} kind (we will come back to this later)
        \end{itemize}
        \item the examples
        \begin{itemize}
            \item use TypeScript syntax (easily readable)
            \item omit the \texttt{id} attribute
        \end{itemize}
    \end{itemize}
\end{frame}


\begin{frame}[fragile]
    \frametitle{Object Constructor - OMOBJ}
    \begin{itemize}
        \item
\begin{lstlisting}
{
    "kind": "OMOBJ",
    /** optional version of openmath being used */
    "openmath": "2.0",
    /** the actual object */
    "object": omel /* any element */
}
\end{lstlisting}
        \item e.g. the number $3$
\begin{lstlisting}
{
    "kind": "OMOBJ",
    "openmath": "2.0",
    "object": {
        "kind": "OMI", 
        "integer": 3
    }
}
\end{lstlisting}
    \end{itemize}
\end{frame}

\begin{frame}[fragile]
    \frametitle{Symbols - OMS}
    \begin{itemize}
        \item
\begin{lstlisting}
{
    "kind": "OMS",
    /** the base for the cd, optional */
    "cdbase": uri, /* any valid URI */, 
    /** content dictonary the symbol is in, any uri */
    "cd": uri,
    /** name of the symbol */
    "name": name /* any valid symbol name */
}
\end{lstlisting}
        \item e.g. the \texttt{sin} symbol from the \texttt{transc1} CD
\begin{lstlisting}
{
    "kind": "OMS",
    "cd": "transc1",
    "name": "sin"
}
\end{lstlisting}
    \end{itemize}
\end{frame}

\begin{frame}[fragile]
    \frametitle{Variables - OMV}
\begin{itemize}
        \item
\begin{lstlisting}
{
    "kind": "OMV",
    /** name of the variable */
    "name": name
}
\end{lstlisting}
        \item e.g. the variable $x$
\begin{lstlisting}
{
    "kind": "OMV",
    "name": "x"
}
\end{lstlisting}
    \end{itemize}
\end{frame}

\begin{frame}[fragile]
    \frametitle{Integers - OMI (1)}
    \begin{itemize}
        \item integers can be represented in three ways
        \begin{itemize}
            \item as a native JSON integer
            \item as a decimal-encoded string (like in XML)
            \item as a hexadecimal-encoded string (like in XML)
        \end{itemize}
        \item 
\begin{lstlisting}
{
    "kind": "OMI",
    //
    // exactly one of the following
    //

    /* any json integer */
    "integer": integer,
    /* any string matching ^-?[0-9]+$ */
    "decimal": decimalInteger,
    /* any string matching ^-?x[0-9A-F]+.$ */
    "hexadecimal": hexInteger
}
\end{lstlisting}
    \end{itemize}
\end{frame}

\begin{frame}[fragile]
    \frametitle{Integers - OMI (2)}
    \begin{itemize}
    \item e.g. $-120$ represented in three ways:
    \begin{itemize}
        \item as a JSON integer
\begin{lstlisting}
{
    "kind": "OMI",
    "integer": -120
}
\end{lstlisting}
        \item as a decimal-encoded string
\begin{lstlisting}
{
    "kind": "OMI",
    "decimal": "-120"
}
\end{lstlisting}
        \item as a hexadecimal-encoded string
\begin{lstlisting}
{
    "kind": "OMI",
    "hexadecimal": "-x78"
}
\end{lstlisting}
        \end{itemize}
    \end{itemize}
\end{frame}

\begin{frame}[fragile]
    \frametitle{Floats - OMF (1)}
    \begin{itemize}
        \item floats can also be represented in three ways
        \begin{itemize}
            \item as a native JSON number
            \item using their decimal encoding (like in XML)
            \item using their hexadecimal encoding (like in XML)
        \end{itemize}
        \item
\begin{lstlisting}
{
    "kind": "OMF",

    //
    // exactly one of the following
    //

    /* any json number */
    "float": float,
    /* any string matching 
        ^(-?)([0-9]+)?("."[0-9]+)?([eE](-?)[0-9]+)?$ */
    "decimal": decimalFloat,
    /* any string matching ^([0-9A-F]+)$ */
    "hexadecimal": hexFloat 
}
\end{lstlisting}
    \end{itemize}
\end{frame}

\begin{frame}[fragile]
    \frametitle{Floats - OMF (2)}
    \begin{itemize}
    \item e.g. $10^{-10}$ represented in three ways:
    \begin{itemize}
        \item as a JSON float
\begin{lstlisting}
{
    "kind": "OMF",
    "float": 1e-10
}
\end{lstlisting}
        \item as a decimal-encoded string
\begin{lstlisting}
{
    "kind": "OMF",
    "decimal": "0.0000000001"
}
\end{lstlisting}
        \item as a hexadecimal-encoded string
\begin{lstlisting}
{
    "kind": "OMF",
    "hexaecimal": "3DDB7CDFD9D7BDBB"
}
\end{lstlisting}
        \end{itemize}
    \end{itemize}
\end{frame}

\begin{frame}[fragile]
    \frametitle{Bytes - OMB (1)}
    \begin{itemize}
        \item bytes can be represented in two ways
        \begin{itemize}
            \item as an array of bytes
            \item as a string encoded in base64
        \end{itemize}
        \item
\begin{lstlisting}
{
    "kind": "OMB",
    //
    // exactly one of the following
    //

    /** an array of bytes
        where a byte is an integer from 0 to 255 */
    "bytes": byte[],
    /** a base64 encoded string */
    "base64": base64string
}
\end{lstlisting}
    \end{itemize}
\end{frame}


\begin{frame}[fragile]
    \frametitle{Bytes - OMB (2)}
    \begin{itemize}
    \item e.g. the ascii bytes of \textit{hello world} represented in two ways:
    \begin{itemize}
            \item as a byte array
\begin{lstlisting}
{
    "kind": "OMB",
    "bytes": [
        104, 101, 108, 108, 111, 32, 
        119, 111, 114, 108, 100
    ]
}
\end{lstlisting}
            \item as a base64-encoded string
\begin{lstlisting}
{
    "kind": "OMB",
    "base64": "aGVsbG8gd29ybGQ="
}
\end{lstlisting}
        \end{itemize}
    \end{itemize}
\end{frame}

\begin{frame}[fragile]
    \frametitle{Strings - OMSTR}
    \begin{itemize}
        \item
\begin{lstlisting}
{
    "kind": "OMSTR", 
    /** the string */
    "string": string
}
\end{lstlisting}
        \item e.g. \begin{lstlisting}
{
    "kind": "OMSTR", 
    "string": "Hello world"
}
\end{lstlisting}
    \end{itemize}
\end{frame}

\begin{frame}[fragile]
    \frametitle{Applications - OMA (1)}
    \begin{itemize}
        \item
\begin{lstlisting}
{
    "kind": "OMA", 
    /** the base for the cd, optional */
    "cdbase": uri, 
    /** the term that is being applied */
    "applicant": omel, 
    /** the arguments that the applicant 
        is being applied to. Optional and
        assumed to be empty if omitted */
    "arguments"?: omel[]
}
\end{lstlisting}
    \end{itemize}
\end{frame}

\begin{frame}[fragile]
    \frametitle{Applications - OMA (2)}
    \begin{itemize}
        \item e.g. $\sin(x)$ \begin{lstlisting}
{
    "kind": "OMA",
    "applicant": {
        "kind": "OMS",
        "cd": "transc1",
        "name": "sin"
    },
    "arguments": [{
        "kind": "OMV",
        "name": "x"
    }]
}
\end{lstlisting}
    \end{itemize}
\end{frame}

\begin{frame}[fragile]
    \frametitle{Attributions - OMATTR (1)}
    \begin{itemize}
        \item
        \begin{lstlisting}
{
    "kind": "OMATTR", 
    /** the base for the cd, optional */
    "cdbase": uri, 
    /** attributes attributed to this object, non-empty */
    "attributes": ([
        OMS, omel|OMFOREIGN
    ])[],
    /** object that is being attributed */
    "object": omel
}
    \end{lstlisting}
        \item attributes are represented as an array of pairs containing
        \begin{itemize}
            \item the name of the attribute
            \item the value of the attribute
        \end{itemize}
    \end{itemize}
\end{frame}

\begin{frame}[fragile]
    \frametitle{Attributions - OMATTR (2)}
    \begin{itemize}
        \item e.g. to annotate a variable $x$ as having a real type
\begin{lstlisting}
{
    "kind": "OMATTR",
    "attributes": [
        [
            { "kind": "OMS", "cd": "ecc", "name": "type" },
            { "kind": "OMS", "cd": "ecc", "name": "real" }
        ]
    ],
    "object": {
        "kind": "OMV",
        "name": "x"
    }
}
\end{lstlisting}
    \end{itemize}
\end{frame}

\begin{frame}[fragile]
    \frametitle{Bindings - OMB (1)}
    \begin{itemize}
        \item
\begin{lstlisting}
{
    "kind": "OMBIND", 
    /** the base for the cd, optional */
    "cdbase": uri, 
    /** the binder being used */
    "binder": omel,
    /** the variables being bound, non-empty */
    "variables": (OMV | attvar)[],
    /** the object that is being bound */
    "object": omel
}
    \end{lstlisting}
        \item variables being attributed are represented as a list with each element either
        \begin{itemize}
            \item an \texttt{OMV} variable
            \item an \texttt{OMATTR} where the attributed object is a variable (\texttt{attvar})
        \end{itemize}
    \end{itemize}
\end{frame}

\begin{frame}[fragile]
    \frametitle{Bindings - OMB (2)}
    \begin{itemize}
        \item e.g. $\lambda x . \sin(x)$
\begin{lstlisting}
{  
    "kind": "OMBIND",
    "binder": 
        { "kind": "OMS", "cd": "fns1", "name": "lambda" },
    "variables": [  
        { "kind": "OMV", "name": "x" }
    ],
    "object": {  
        "kind": "OMA",
        "applicant":
            { "kind": "OMS", "cd": "transc1", "name":"sin" },
        "arguments": [
            { "kind": "OMV", "name": "x" }
        ]
    }
}
\end{lstlisting}
    \end{itemize}
\end{frame}

\begin{frame}[fragile]
    \frametitle{Errors - OME (1)}
    \begin{itemize}
        \item
\begin{lstlisting}
{
    "kind": "OME", 
    /** the error that has occured */
    "error": OMS,
    /** arguments to the error, optional  */
    "arguments"?: (omel|OMFOREIGN)[]
}
    \end{lstlisting}
    \end{itemize}
\end{frame}

\begin{frame}[fragile]
    \frametitle{Errors - OME (2)}
    \begin{itemize}
        \item e.g. to annotate a \textit{division by zero} error in $x / 0$
\begin{lstlisting}
{
    "kind": "OME",
    "error":
        { "kind": "OMS", "cd": "aritherror", 
          "name": "DivisionByZero" },
    "arguments": [{
        "kind": "OMA",
        "applicant": { "kind": "OMS", "cd": "arith1",
                       "name": "divide" },
        "arguments": [
            { "kind": "OMV", "name": "x" },
            { "kind": "OMI", "integer": 0 }
        ]
    }]
}
\end{lstlisting}
    \end{itemize}
\end{frame}

\begin{frame}[fragile]
    \frametitle{Foreign Objects - OMFOREIGN}
    \begin{itemize}
        \item
\begin{lstlisting}
{
    "kind": "OMFOREIGN",
    /** encoding of the foreign object, optional */
    "encoding"?: string,
    /** the foreign object */
    "foreign": any
}
\end{lstlisting}
        \item e.g. to represent a latex math term $\sin(x)$
\begin{lstlisting}
{
    "kind": "OMFOREIGN",
    "encoding": "text/x-latex",
    "foreign": "$\sin(x)$"
}
\end{lstlisting}
    \end{itemize}
\end{frame}


\begin{frame}[fragile]
    \frametitle{References - OMR (1)}
    \begin{itemize}
        \item we can reference any object with an \texttt{id}
        \item
\begin{lstlisting}
{
    "kind": "OMR"
    /** element that is being referenced */
    "href": uri
}
\end{lstlisting}
        \item e.g. the term $f(\underbracket {f(\overbracket {f(a, a)}^y, y)}_x, x)$
    \end{itemize}
\end{frame}

\begin{frame}[fragile]
    \frametitle{References - OMR (2)}
\begin{lstlisting}
{
    "kind": "OMOBJ",
    "object": {
        "kind": "OMA",
        "applicant": { "kind": "OMV", "name": "f" },
        "arguments": [{ 
            "kind": "OMA", "id": "x",
            "applicant": { "kind": "OMV", "name": "f" },
            "arguments": [{
                "kind": "OMA", "id": "y",
                "applicant": { "kind": "OMV", "name": "f" },
                "arguments": 
                    [{ "kind": "OMV", "name": "a" },
                    { "kind": "OMV", "name": "a" }]
            }, { "kind": "OMR", "href": "#y" }]
        }, {
            "kind": "OMR", "href": "#x" 
        }]
    }
}
\end{lstlisting}
\end{frame}

% =================================================================
% Future Work: SCSCP based on the OpenMath encoding
% =================================================================
% TODO: Write these slides

% =================================================================
% Summary
% =================================================================

\begin{frame}
    \frametitle{Summary}
    \begin{itemize}
        \item we established that an OpenMath JSON encoding makes using OM much easier in many languages
        \begin{itemize}
            \item most languages have strutured data types built in
            \item serialization into/from JSON exists natively in many languages
            \item easy to make use of \textit{Protocol Buffers} or \textit{ZeroMQ} based on this work
        \end{itemize}
        \item existing approaches had disadvantages, so we developed our own
        \begin{itemize}
            \item simple to translate to/from the XML Encoding (we have built a translator)
            \item uses JSON-native data types
        \end{itemize}
        %\item can be used for protocols based on OpenMath, such as SCSCP
        %\begin{itemize}
        %    \item some future work still needed for this
        %\end{itemize}
        \item Thank you for listening. Questions, Comments, Concerns?
    \end{itemize}
\end{frame}

\end{document}
